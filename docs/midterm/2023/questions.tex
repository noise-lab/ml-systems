\section*{Use Cases}
\vspace*{-0.1in}
\prob{4} Machine learning has become a useful tool in modeling and predicting
the performance of networked systems. In class and in the assignments, you
studied how machine learning can be used to predict various aspects of
application quality. For the case of {\em video on demand}, give an example of
an (1) {\bf three} application quality metrics that machine learning is often used to
predict; (2) {\bf one} feature that is often used to predict these metrics.
\\
\answerbox{1}{Various application quality metrics include: frame rate,
resolution, and startup delay. Features often used to predict these
application quality metrics include segment rate, throughput/bitrate, and
latency. }
\eprob

\prob{4} Internet scanning is a technique to discover vulnerabilities of
machines that are connected to the Internet. One form of scanning that we
studied was HTTP (Web) scanning, as in the Log4j vulnerability scan. As we
studied, the HTTP traffic generated by the scan looked ``different'' than
regular Web traffic. Give two features that could be used to distinguish
between HTTP scanning traffic and regular Web traffic.
\\
\answerbox{1}{ Possible features (including many suggested by you in class!)
include: (1) the total number of requests per second, (2) the number of unique
requests per second, (3) the total number of bytes transferred; (4) the
duration of the connection. Many other possibilities exist.}
\eprob

\section*{Data Acquisition}

\prob{2} 
    With access to unencrypted DNS queries and responses and knowledge about
    which DNS names correspond to a particular service (e.g., Netflix), it is possible to extract
    IP addresses for that service.
\framebox{
\yesnoyes
}
\eprob
Alyssa P. Hacker wants to extract all Netflix traffic from packet traces, but
reasoning that the code for matching DNS names to IP addresses is cumbersome,
she decides instead to simply hard-code the IP addresses for a particular
service. Alyssa had constructed her initial IP address list by looking up
domain names for Netflix here in Chicago.  
When she deploys her code to networks in other locations around the world, she
is surprised to find that her code is not extracting any Netflix traffic!

\prob{2}
Give one reason why Alyssa's code may not be capturing Netflix traffic in
other locations.
\\
\answerbox{1}{
Applications serve traffic from content
delivery networks (CDNs), with servers that are close to users. Therefore,
clients in different cities will receive traffic from different servers, with
different IP addresses.
}
\eprob

\prob{3}
List three advantages to passive Internet measurement over active Internet measurement.
\\
\answerbox{1}{
    No requirement for specific server (``target'') deployment, no need to
    develop bespoke tools for different services/targets of measurement, less
    resource intensive.
}
\eprob


\section*{Feature Engineering}

\if 0
In class, we looked at some cases where ML models perform sub-optimally when
trained exclusively with raw data and  
that {\em domain knowledge} is often useful for feature engineering.
\fi
\prob{2}
Explain {\bf one advantage} of using domain knowledge to engineer features for
model input as opposed to using byte representations of traffic as input.
(There are several correct answers.)
\\
\answerbox{1}{ There are several possible answers, including avoiding spurious
correlations in data, reducing the dimensionality of the data, and producing
models whose decisions are easier to understand. }
\eprob
\prob{5}
The University of Chicago ITS wants to identify compromised devices on its
network and provides you with a labeled traffic trace of benign and
compromised devices to build a model to detect them. Your model achieves 99\%
accuracy on the training data, and the most important feature is the set of
domain names that each device contacts. Your model works well on the training
set and for the first few days, but a week later, it is failing to detect many
of the compromised devices. (1)~Why might this feature have worked initially?
(2)~Why might it have stopped working?
\\
\answerbox{0.75}{ (1) Compromised devices often contact strange domain names.
(2) These domains can change over time, as malicious infrastructure changes,
or as different malware/compromises become prevalent. }
\eprob

\section*{Feedback}
\vspace*{-0.1in}
\prob{1}
Interest (1=Boring!; 10=Amazing!):
\shortanswerbox{0.5}{5}
Difficulty (1=Too easy; 10=Too hard):
\shortanswerbox{0.5}{5}
\eprob
\prob{1}
1. One thing you like. 2. One suggestion for improvement:

\answerbox{0.75}{More free food.}
\eprob

